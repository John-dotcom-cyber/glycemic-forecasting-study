\chapter{Introduction}

La gestion de la glycémie en réanimation constitue un enjeu majeur pour la prise en charge des patients critiques. 
Les variations glycémiques, en particulier l’hyperglycémie persistante ou les fluctuations importantes, sont associées à une augmentation de la morbidité, du risque infectieux et de la mortalité. Dans ce contexte, la capacité à caractériser précisément les profils glycémiques des patients et à identifier ceux présentant une sévérité accrue représente un objectif clinique essentiel.

Ce projet s’inscrit dans cette perspective. À partir de données de glycémie mesurées en unité de soins intensifs (ICU), l’objectif est de construire un pipeline complet d’analyse permettant :

\begin{itemize}
    \item de transformer des données temporelles brutes en \textbf{features cliniquement pertinentes} au niveau patient,
    \item d’explorer la structure des données via des méthodes non supervisées (corrélations, clustering hiérarchique, PCA),
    \item de définir un \textbf{label de sévérité glycémique} cohérent avec la littérature clinique,
    \item et d’évaluer la capacité de différents modèles supervisés à prédire cette sévérité.
\end{itemize}

Bien que le dataset étudié soit de taille réduite (10 patients), il permet d’illustrer de manière rigoureuse l’ensemble des étapes d’un projet de data science clinique : nettoyage, feature engineering, analyse exploratoire, modélisation et interprétation. 
L’objectif n’est pas de produire un modèle généralisable à grande échelle, mais de démontrer la cohérence méthodologique et la pertinence clinique d’un pipeline reproductible.