\chapter{Résultats}

\section{Performances des modèles}

Le Tableau~\ref{tab:perf} présente les performances obtenues par les différents modèles.

\begin{table}[H]
\centering
\begin{tabular}{lccc}
\hline
\textbf{Modèle} & \textbf{AUC} & \textbf{F1-score} & \textbf{Recall} \\
\hline
Régression logistique & 1.00 & 1.00 & 1.00 \\
Random Forest 1.00 & 1.00 & 1.00 & 1.00 \\
Gradient Boosting 1.00 & 1.00 & 1.00 & 1.00 \\
\hline
\end{tabular}
\caption{Performances des modèles de classification.}
\label{tab:perf}
\end{table}

\section{Courbes ROC}

\begin{figure}[H]
    \centering
    \includegraphics[width=0.75\textwidth]{images/roc_smote.png}
    \caption{Courbes ROC des différents modèles.}
    \label{fig:roc}
\end{figure}

\section{Importance des variables}

\begin{figure}[H]
    \centering
    \includegraphics[width=0.75\textwidth]{images/feature_importance.png}
    \caption{Importance des variables selon le modèle Random Forest.}
    \label{fig:importance}
\end{figure}
