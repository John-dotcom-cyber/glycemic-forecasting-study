\chapter{Conclusion}

Ce travail avait pour objectif d’explorer la possibilité de prédire un profil glycémique sévère à partir de mesures continues de glucose. Après un nettoyage approfondi des données et la construction de variables dérivées pertinentes, plusieurs modèles supervisés ont été évalués.

L’analyse exploratoire a mis en évidence une forte variabilité inter-patient et un déséquilibre important entre les classes. L’utilisation de la méthode SMOTE s’est révélée indispensable pour permettre aux modèles d’apprendre efficacement. Les résultats obtenus montrent que des modèles relativement simples, tels que la régression logistique ou le Random Forest, peuvent atteindre d’excellentes performances lorsque les données sont correctement préparées.

Bien que les conclusions soient limitées par la taille réduite du dataset, cette étude démontre la faisabilité d’une approche basée sur des features statistiques simples pour caractériser des profils glycémiques. Des travaux futurs pourraient inclure l’intégration de données temporelles plus riches, l’utilisation de modèles séquentiels ou l’analyse de cohortes plus larges afin d’améliorer la robustesse et la généralisation des prédictions.
